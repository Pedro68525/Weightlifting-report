Taking into account the steps mentioned in chapter \ref{phasesChapter} for the design and development of a product ontology, the first step that should be performed is the definition of the ontology's domain and use cases scenarios. This is about capturing the necessary knowledge for the framework development.
\par The domain defines the knowledge area that the structure and vocabulary of an ontology is designed for.
\par The use cases scenarios are defined in order to acquire the main objectives for the ontology, the set of questions that ontology should answer, the users that interact with it, and the questions related with its maintainability.
\par Sanya and Shehab \cite{AeroArticle} define these terms through four simple questions that should be answered in this phase:
\begin{enumerate}
\item What is the domain?
\item What can the ontology be used for?
\item What questions should be answer?
\item Who will use and maintain it?
\end{enumerate}

\par The ontology developed by Hugo Dias \cite{HugoThesis} in his thesis refers to the biomechanical domain in Olympic Weightlifting training sessions.
This ontology is used to improve the posture of weightlifting athletes, and with this improve their performance and prevent serious injuries. It can also be used to see the progress of an athlete and help managers teams in the training sessions.
It should answer if the athlete, during a training session, performed the weightlifting properly. If not, it should mention what was wrong and what body position should be changed in order to perform the exercise properly.
The developed ontology will be used by athletes and their managers teams. It can be assumed that it will not have any type of maintenance since there's no information about that issue in Hugo Dias' \cite{HugoThesis} thesis. 
\par Table \ref{tab:1st} summarizes all the information that should be collected during this phase. 

\begin{table}[H]
\centering
\caption{Step 1 - Defining the domain and use cases scenarios.}
\label{tab:1st}
\begin{tabular}{p{7cm} p{7cm}}
\hline
What is the domain?                        & \begin{tabular}[c]{p{7cm}} - Biomechanical knowledge in Olympic Weightlifting  \end{tabular}
\\ \hline
What can the ontology be used for?         & \begin{tabular}[c]{p{7cm}}-
 Understanding the failures of athletes \\ - Improving the posture of athletes\\ - Preventing injuries in the trains\\ - Improving the overall performance\\ - Verify the athletes' progress\\  - Help managers and athletes\end{tabular} \\ \hline
What questions should the ontology answer? & \begin{tabular}[c]{p{7cm}}- The athlete performed the weightlifting correctly?\\ - There was something wrong with the athlete's posture?\end{tabular}                                                                                                                                                                   \\ \hline
Who will use and maintain it?              & \begin{tabular}[c]{p{7cm}}- It will be used by athletes and their managers teams.\\ - It won't be maintained.\end{tabular}                                                                                                                                                                        \\ \hline
\end{tabular}
\end{table}